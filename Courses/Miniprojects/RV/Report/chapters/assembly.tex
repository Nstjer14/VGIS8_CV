\chapter{Assembly}
\section{Robot Control}
The robot was controlled by using a Matlab script that creates an interface to the UR robot. This is done with a TCP/IP connection. The robot is then controlled linearly or by joint values. Since the location of the brick is known, inverse kinematics can be used get the robot to that location.  As long as the location and orientation of where the end effector is supposed to be, the internal software of the robot takes care of the inverse kinematics.

When actually performing it in practice this is how we did it. First the hand eye calibration was run. The points chosen for calibration were the bolts that were in the workspace because they were constant points. The robot tool was then placed horizontal to the workspace using the teach pendant so the bolt was in the centre of the gripper. The tool's x and y positions were then saved for calibration. Since these positions would not change as the workspace was where the robot was mounted it was not necessary to get more than once. The the same pixel values of the same location was found and the calibration was done.  

After the calibration a brown flat cardboard was put onto the workspace to minimize the reflection from the metallic surface. Then an image of the workspace is taken without the bricks and with the bricks for background subtraction. Then the image processing is done resulting in brick centres, colours and rotations. The robot can now be ordered to assemble a figure e.g. Homer by taking the first appropriate bricks from the top of the image. It moves to the centre of the brick's position an rotates the corresponding degrees around the Z axis to be able to grab the brick.  The figure is then delivered to the drop off zone and prepares for a new order. 