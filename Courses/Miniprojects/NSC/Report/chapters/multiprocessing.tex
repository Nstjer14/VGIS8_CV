\section{Multiprocessing}
The whole flow of the image processing is sequential and it does not make sense to parallelise as the processes are depended on each other. But since the images are independent from each other, it is a Single Program Multiple Data (SPMD) situation. In other words it does not know what order the images are processed in and the same image processing is applied to all images. There is potential for improvement in performance by utilising more than one CPU core for processing. This is implemented by using the Python package Multiprocessing. Here a pool of workers is created which can work on the data. When a worker has finished it's task it takes the next task in line, and so on. The tasks are scheduled automatically by the package. The function map_async() is used as the work does not need to be done synchronously. map_async().get is also used to make the program wait until the processing is done and return everything at once. The data is then saved into a Pandas dataframe for later usage.