\chapter{Echo Audio Effect Measurements}\label{ch:echo_meas_app}
The echo effect is tested to compare the effect to the requirements and analysis description of the effect.

\section{Purpose}
The impulse response of the echo effect is analysed according to delay and amplitude to see, if the output corresponds to \autoref{req:DL1} and \autoref{req:DL3} and if it behaves as described in \autoref{sec:effect_descr}.

\section{List of Equipment}
\begin{itemize}
	\item Digilent Analog Discovery 2 including BNC-adaptor board
	\item Philips PM 5715 pulse generator (AAUnr. 08644)
	\item TMS320C5515 eZdsp
	\item A computer
\end{itemize}

\section{Set-up}
The pulse generator sends a single impulse into the \gls{dsp}. The output signal from the \gls{dsp} is read on the Digital Analog Discovery 2 and analysed. By using the Waveforms software set at single input, only one impulse is recorded and saved at a time.

\section{Procedure}
An impulse is sent in on the input of the \gls{dsp} and the impulse response is then plotted and analysed according to delay and amplitude.

The impulse is sent with the Philips PM 5715 pulse generator. This is set to do single shot, which is controlled by the user. The duration needed is \SI{1}{\milli\second} to intercept the signal on the oscilloscope. The ramp is set at the lowest possible value of \SI{6}{\nano\second}.

\section{Results}
The echo is generated with delays of $2500, ~4410 ~\text{and} ~29\,999$ samples. The delays of $2500 ~\text{and} ~29\,999$ samples are with a decay of $500$. The delay of $4410$ samples is with decays $500$ and $900$. These four measurements are shown in \autoref{fig:echo_del4410_app} and \autoref{fig:echo_dec500_app}.

\begin{figure}[htbp]
	\centering
	\includegraphics[width=\textwidth]{figures/appendix/echo_plot_del4410}
	\caption{The two plots show the impulse echoed with a delay of $ 4410 $ samples and decays of $ 500 $ and $ 900 $.}
	\label{fig:echo_del4410_app}
\end{figure}

\begin{figure}[htbp]
	\centering
	\includegraphics[width=\textwidth]{figures/appendix/echo_plot_dec500}
	\caption{The two plots show the impulse echoed with a delays of $ 2500 $ and $ 29\,000 $ samples and a decay of $ 500 $.}
	\label{fig:echo_dec500_app}
\end{figure}

The delay can be varied down to 1 sample if needed, but according to \autoref{req:DL1} the delay must not go below \SI{50}{\milli\second}. This means the minimum amount of delay samples is $44100\cdot0.050=2205$. The highest delay possible is $29\,999$ as this is limited by the buffer length. This means that \autoref{req:DL1} is met.

\autoref{req:DL3} requires the echo to have a variable gain of the delayed signals. This is done by adjusting the \textit{decay} factor of the effect. The lower the decay value is the faster the signal will die out. This means that \autoref{req:DL3} is also fulfilled.

The figures also show how the echo effect being played, means a repetition of the original signal at a constant delay, but with a decay to the amplitude.