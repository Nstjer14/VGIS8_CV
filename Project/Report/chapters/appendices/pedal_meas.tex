%!TEX root = ../../master.tex
\chapter{Measuring Signal Through Analogue Distortion Pedal} % (fold)
\label{cha:measuring_distortion_pedal_output}

\section{Purpose}
Because distortion and overdrive are considered to be difficult to recreate digitally a measurement of an analogue distortion pedal is made. The purpose of the measurements is to see what the output is to make it easier to implement digitally. As seen in \autoref{subs:dist_overdrive} and because the measurements are made on the distortion part in a Heavy Deavy Skull Fuzz which is a separate distortion and fuzz pedal build into one case \citep{skull_fuzz}, hard clipping is expected  on the output. Both the amplitude and the frequency response is measured at different gain settings on the pedal. 

\section{List of Equipment}
\paragraph*{Amplitude}

\begin{itemize}
	\item Signal generator
	\item Heavy Deavy Skull Fuzz \citep{skull_fuzz}
	\item Oscilloscope 
	\item Speaker
\end{itemize}

\paragraph*{Frequency Response}

\begin{itemize}
	\item Heavy Deavy Skull Fuzz \citep{skull_fuzz}
	\item NI-4461 - sine sweep computer
\end{itemize}

\section{Set-up}
\paragraph*{Amplitude}

A signal generator is connected to the input of the pedal and on the output of the pedal there is connected either a speaker or an oscilloscope depending on which test is made as seen in \autoref{fig:heavy_deavy_meas_block}. There is made both a listening test and a test where the signal is measured on the oscilloscope.

\begin{figure}[htbp]
	\centering
	\includegraphics[scale=1]{figures/appendix/pedal_measure.pdf}
	\caption{Setup with possibility of switching between listening to the signal through a speaker or measuring with an oscilloscope.}
	\label{fig:heavy_deavy_meas_block}
\end{figure}


\paragraph*{Frequency Response}
The pedal is connected to the NI-4461 from $V_{Ao0}$ to the pedals input and from the pedals output to $V_{Ai1}$. To have a reference when the frequency response is measured $V_{Ao0}$ is connected to $V_{Ai0}$ on the NI-4461 as seen in \autoref{fig:heavy_deavy_freqresp_block}.


\begin{figure}[htbp]
	\centering
	\includegraphics[scale=0.6]{figures/appendix/pedal_freqresp.pdf}
	\caption{Setup for measuring the frequency response of the effect pedal.}
	\label{fig:heavy_deavy_freqresp_block}
\end{figure}



\section{Procedure}

\paragraph*{Amplitude}

The signal generator is set to 440 Hz with an amplitude of 0.5 V and the volume and tone knob on the pedal is set to 0.5 of their maximum as seen on \autoref{fig:skull_fuzz_settings}.

\begin{figure}[H]
	\centering
	\includegraphics[width=0.4\textwidth]{figures/appendix/skull_fuzz_settings.png}
	\caption{Picture of the Heavy Deavy Skull Fuzz showing the distortion part of the pedal (green) and the five different gain settings where 1 = 33 \%, 2 = 50 \% and 3 = 66 \% of the maximum gain.}
	\label{fig:skull_fuzz_settings}
\end{figure}

The gain is regulated for each test at the values minimum, 33 \%, 50 \%, 66 \% and maximum also seen in \autoref{fig:skull_fuzz_settings}. To make sure what to expect when doing the measurements with the oscilloscope, a test is made by changing the gain of the pedal while listening to the signal through a speaker as seen as an option on \autoref{fig:heavy_deavy_meas_block}. By doing this an expectation for when the clipping starts is made. When the listening test is made the pedal is connected to an oscilloscope and the same signal is send through the pedal at the five different gain settings. 

\paragraph*{Frequency Response}

The pedal is connected to the computer as seen in \autoref{fig:heavy_deavy_freqresp_block}. The computer then plays a \SI{1}{\volt} sinus sweep from \SI{20}{\hertz} to \SI{20}{\kilo \hertz} from the output channel $V_{Ao0}$. The signal is measured directly from $V_{Ao0}$ by the input $V_{Ai0}$ and through the pedal by the input $A_{Ai1}$.The two signals are compared to show the frequency response through the pedal. This procedure is repeated at the three different gain settings.


\section{Results}
\paragraph*{Amplitude}
On \autoref{fig:skull_fuzz_output_voltage} the output from the pedal in volts is seen. With the gain knob at minimum the amplitude has a peak value of 0.37 V and with the gain at maximum the amplitude is peaking at 0.75 V. The ratio between these is approximately two which is equal to 6 dB. 

\begin{figure}[htbp]
	\centering
	\includegraphics[scale=0.8]{figures/appendix/Distortion_plot.eps}
	\caption{Output from the pedal in volts}
	\label{fig:skull_fuzz_output_voltage}
\end{figure}


\paragraph*{Frequency Response}
The frequency response of the pedal is shown in \autoref{fig:skull_fuzz_output_freq_resp}. It is seen that the amplification at 440 Hz, which is the same frequency as the other measurement is made, is $1.2\text{ dB} + 5.6\text{ dB} = 6.8\text{ dB}$. The reason for the different amplification between the measurements is caused by small changes on the pedal knobs but the difference is only 0.06 V which is acceptable. 

\begin{figure}[htbp]
	\centering
	\includegraphics[scale=0.8]{figures/appendix/distortion_freqresp.eps}
	\caption{Graph of the frequency response of the Heavy Deavy Skull Fuzz.}
	\label{fig:skull_fuzz_output_freq_resp}
\end{figure}

The frequency response is showing a high pass and a low pass filter in the pedal with a 3 dB drop a 22 Hz and 5000 Hz.

As seen in \autoref{fig:skull_fuzz_output_voltage} the amplitude is amplified by turning up the gain and at approximately 33 \% of the way up when the distortion is starting it is seen that the signal is starting to clip. From 33 \% and up the signal is becoming more and more square until the maximum gain level is reached with a maximum amplification of 6.8 dB with the volume knob half way up. The measurements show the same as expected, but the conclusion is that the clipping must happen soft at first and then harder the more gain is applied.

The measurements of the frequency response verifies the amplification and shows a \SI{3}{\decibel} attenuation of the signal at \SI{22}{\hertz} and \SI{5000}{\hertz}. This means that when designing the digital effects it is acceptable only to work in the audible frequency spectrum which is from \SI{20}{\hertz} to \SI{20}{\kilo \hertz}. 