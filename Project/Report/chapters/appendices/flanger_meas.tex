%!TEX root = ../../master.tex
\chapter{Flanger Audio Effect Measurements}\label{flanger_meas_app} 
Flanger is tested with regard to \autoref{req:DL2} and compared to the designed flanger in \autoref{sec:flanger_design}.
\section{Purpose}
How the frequency spectrum changes over time is analysed in order to see if the delayed signal goes in and out of phase with itself generating peaks and troughs moving up and down the frequency spectrum according to \autoref{sec:flanger_design}.
\section{List of Equipment}

\begin{itemize}
	\item A computer with a sound card
	\item TMS320C5515 eZdsp
\end{itemize}

\section{Set-up}
The setup for the test is shown in the block diagram in \autoref{fig:flanger_block_setup}
\begin{figure}[H]
	\centering
	\includegraphics[scale=0.8]{figures/appendix/flanger_setup}
	\caption{Test setup of flanger.}
	\label{fig:flanger_block_setup}
\end{figure}
The input of the \gls{dsp} is coupled up to the output of the computer with a mini-jack audio cable. The same is done from the \gls{dsp} output to the computer's input. A file with white noise is played into the \gls{dsp} and the returned data is recorded and analysed in MATLAB. 
\section{Procedure}

White noise is played from the computer into the \gls{dsp} and the returned data is then recorded on the computer. The recorded audio data is analysed in MATLAB by plotting the spectrogram.

\section{Results}
The resulting data from the test is shown on the spectogram in \autoref{fig:test_flanger}. 
\begin{figure}[hbpt]
	\centering
	\includegraphics[width=\textwidth]{test_flanger}
	\caption{Spectrogram of white noise sent through flanger.}
	\label{fig:test_flanger}
\end{figure}

It is seen, that the spectogram has the same kind of troughs and spikes as the ones shown in \autoref{fig:freq_flanger}. When comparing the simulated graph and the graph from the test, it can be seen, that they yield the same results, and thus the test is evaluated as successfull 
