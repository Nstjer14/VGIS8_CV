The problem of protection of physical elements or information by limiting the access to only the people  is well known. The central challenge to the problem of security is the challenge of verifying the identity of a person trying to acquire access. The emergence of biometric techniques has induced an increasing interest in biometric-based security rather than knowledge-based or token-based security. This especially because the more traditional methods for security systems are easier breached or spoofed \citep{Ross2003}. Through the last decades researchers have investigated identity verification based on different biometric modalities. In the last decade investigations have been conducted in combining several biometric modalities in one system with the purpose of creating a system that performs better than the ones only utilising one modality. Research has shown that the combining og modalities performs better than any of the modalities alone \citep{Chen2005a}.  However, increased accuracy is not the only benefit of utilising multiple biometric traits. More modalities increases the universality of the system and decreases the influence of noisy measurements\citep{Ross2003}.

One of the fields where biometric-base security is increasingly applied is security for mobile devices. Though technology is advancing and mobile devices are equipped with still more advanced components, the limited computing power and the limitations in the kind and the quality of the data that can be acquired by the sensors, are still limiting factors. This makes it even more challenging to make successful biometric-based identity verification on mobile devices.\todo{add source?}

The work described in this report strives to make a system for identity verification based on multiple biometric traits for use on mobile devices. The biometric traits used are the iris and face. This choice is made due to the arguments found in literature that iris is very distinctive, while face is non-invasive.  \todo{add source}