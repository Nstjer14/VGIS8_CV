\section{Digital Signal Processor}\label{sec:dsp_board}
In signal processing all audio signals, such as the signal from an electric guitar, are analogue by nature. In digital signal processing, the analogue signal is represented by a sequence of finite-precision numbers, and the signal is then processed using digital computations. The main difference between an analogue signal and a digital signal is, that the analogue signal is a continuous time and continuous amplitude signal, while the digital signal is a discrete time and discrete amplitude signal. 
Using a digital signal processing system has many advantages, such as not being affected by ageing of components or the working environment. These systems are precise and they make advanced operations possible, which would otherwise be hard to realize in an analogue system \citep{kuulusa}. 

The main component of digital processing is a \glsentryfull{dsp}. This can be compared to a standard microprocessor, but it has been specially developed for signal processing, and therefore among other things it has fast multiply-add or multiply-accumulate-operations \citep{kuulusa}. The effects designed in this section will later be implemented on such a \gls{dsp}, in this case the TMS320C5515, sitting on the Digital Spectrum eZdsp development board, which is shown in \autoref{fig:board_dsp}. This \gls{dsp} has been chosen, since it is the processor used in a semester course related to the project.

\begin{figure}[H]
	\centering
	\includegraphics[width=0.7\textwidth]{board_dsp}
	\caption{The Digital Spectrum eZdsp development board}
	\label{fig:board_dsp}
\end{figure}


\subsection{TMS320C5515}
The TMS320C5515 is a fixed-point \gls{dsp}, which is designed for low-power applications. The \gls{dsp} is build around the TMS320C55xx \gls{dsp} generation \gls{cpu} processor core. The architecture of the chip makes it possible to reach high performance and low power consumption through the use of parallelism and a focus on power savings. 

The \gls{cpu} has an internal bus structure which consist of the following buses: one program bus, one 32-bit data read bus, and two 16-bit data read buses. Additionally there are two 16-bit data write buses, and some buses dedicated to peripheral and \gls{dma} activity. This structure makes it possible to perform up to four 16-bit data reads and two 16-bit data writes in one cycle. Four \gls{dma} controllers are also included in the device, that each have four channels. This makes it possible to have data movement in 16 independent channel contexts without the \gls{cpu} intervening. Each of the \gls{dma} controllers can perform one 32-bit data transfer for each cycle, which is parallel and independent of activity of the \gls{cpu}.

The C5515 \gls{cpu} has two \gls{mac} units, that each can handle 17-bit times 17-bit multiplication, they can also add 32-bit numbers in a single cycle. The main 40-bit \gls{alu} is supported by an additional 16-bit \gls{alu}. When using the \gls{alu}s it is under instruction set control, which makes parallel activity possible which provides low power consumption. These resources are managed in the \gls{au} and \gls{du} of the C5515 \gls{cpu} \citep{dsp}.

\subsection{eZdsp development board}
Other than the \gls{dsp}, the Digital Spectrum eZdsp development board also includes several other components. The following components are the ones used in the report. One of the main components is the AIC3204 audio \gls{codec} located at the audio input/output on the board. This audio \gls{codec}  includes a build in \gls{adc} and a \gls{dac} \citep{dspadc}. The board also includes a \gls{oled} display and two buttons for user interaction \citep{ezdsp}.










