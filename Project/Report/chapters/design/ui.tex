\section{Interface}\label{sec:ui_design}
According to \autoref{sec:reqUI} a \gls{ui} must be implemented. This is done by using the \gls{oled} screen included on the TMS320C5515 eZdsp board as the menu monitor and by including four navigation buttons.

\subsection{Physical Interface}\label{sec:physUI}
According to \autoref{req:UIphys} the physical part of the \gls{ui} must have four physical buttons. These are used with the \gls{led} screen included in the TMS320C5515.

The buttons are implemented via the \gls{gpio} pin-out on the development board. The purpose of the buttons so to ease the use of the effects for the user, and will enable the user to choose which effects to activate or deactivate. The configuration of the four buttons is up- and down buttons and an \textit{enter}- and a \textit{back} button. This allows the user to navigate an interface menu and adjust the effects to the desired parameters. An illustration of the four buttons is shown in \autoref{fig:ui_butt}.

\begin{figure}[htbp]
	\centering
	\includegraphics[width=0.6\textwidth]{physUI}
	\caption{Illustration showing the layout of the four buttons of the \gls{ui} and their functions.}
	\label{fig:ui_butt}
\end{figure}

The TMS320C5515 eZdsp board has two buttons included on the board. These are connected to the \gls{sar} \gls{adc}. By doing so, several buttons can be connected to a single pin, as the \gls{adc} reads the analogue value of the signal and converts this to a digital signal.

The pin used for this is the GPAIN1 which is also located on the \gls{gpio} pin-out on pin number 54. This leads into channel three on the \gls{sar} \gls{adc} \citep{dsp_sar}.

To make the conversion in the \gls{adc} a reference voltage is needed. This is set by setting in the \gls{sar} A/D reference and PIN Control Register shortened to SARPINCTRL. From there channel 3 must be selected, which is done in the SARCTRL register by writing a 3 in hex in bit 14-12 \citep{dsp_sar}. 

\subsection{User Interface}
To make a \gls{ui} a visual interface is essential. This will enable the user to navigate a menu and select the wanted effect and set the parameters for it. The navigation of the menu is done by using the physical interface described in \autoref{sec:physUI}.

The display is an OLED 96x16p display. It has two data lines which means there will be two lines of 96x8p each. This is also shown in \autoref{fig:display_distr}.

\begin{figure}[htbp]
	\centering
	\includegraphics[width=0.6\textwidth]{display_distribution}
	\caption{Overview of the display showing distribution of the two data lines.}
	\label{fig:display_distr}
\end{figure}

When writing on the display each pixel on the board must be set individually to light up or not. To do this a bitmap of each letter and the number from 0 to 9 are made. En example of this is shown in \autoref{fig:bitmap}.

\begin{figure}[htbp]
	\centering
	\includegraphics[width=0.7\textwidth]{bitmap}
	\caption{Example of the bitmap for the word \textit{ECHO}. With the hex value for each column.}
	\label{fig:bitmap}
\end{figure}

As shown on \autoref{fig:bitmap} a space is left open on both the right side and in the bottom of the bitmap. This is done to space the letters from each other avoiding them from melting together.

To establish communication between the \gls{dsp} and the display, the included \gls{i2c} connection is used.

\subsubsection{Direct Memory Access}
The \gls{dma} is used to move data between internal memory, external memory and peripherals independently of the \gls{cpu} \citep{dsp_dma}. This means the \gls{cpu} will be able to run a main program meanwhile the \gls{dma} will run in parallel.

As \gls{i2c} is used to communicate with the display the only \gls{dma} controller usable is controller two, since this is the only \gls{dma} controller connected to the \gls{i2c} bus\citep{dsp_dma}.

The \gls{dma} uses byte addresses whereas the \gls{cpu} uses word addresses. This means, to generate \gls{dma} addresses an offset must be added to the \gls{cpu} addresses. This is done by bit shifting the word address one bit to the left to create a byte address. If DARAM or SARAM is accessed an offset correction is needed as well. This is done by adding \texttt{0x10000} or \texttt{0x80000} respectively to the address.


To set up the \gls{dma} transferring the \textit{Transfer Control Register 2} is used. An overview of this \SI{16}{\bit} register is shown in \autoref{fig:dma_transCR2}.

\begin{figure}[htbp]
	\centering
	\includegraphics[width=0.9\textwidth]{dma_transCR2}
	\caption{Overview of the \SI{16}{\bit} \textit{Transfer Control Register 2} from \cite{dsp_dma}.}
	\label{fig:dma_transCR2}
\end{figure} 
