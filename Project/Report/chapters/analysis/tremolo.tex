\subsection{Tremolo}

\todo[author=Jacob]{Skal nok slettes}
The tremolo is as stated in \autoref{sec:problem_analysis} the first effect pedal made. The tremolo effect sounds as if the volume is increased and decreased with a specific rate. Most often the pedals have the options to change the rate or the speed of the effect varying how fast the amplitude is changing, as well as changing the depth of the effect. This means the user can change the speed and the volume of the effect \citep{tremolo_descr}. A simple block diagram of the tremolo effect with these properties is shown in \autoref{fig:tremolo_block}.

\begin{figure}[htbp]
	\centering
	\includegraphics[width=0.8\textwidth]{tremolo_block}
	\caption{Simple example of a tremolo effect with two internal configurations.}
	\label{fig:tremolo_block}
\end{figure}

There are a few ways to realise this effect, but there are no specific guidelines as to how high the rate of the effect should be, and also what the depth should be. In the analogue world a tremolo effect is usually realised with an \gls{lfo} combined with both a high pass and a low pass filter. The general idea behind the \gls{lfo} design is, that the phase of the signal is changed, before it reaches the amplifier and thus giving it the desired tremolo effect. A lot of effects also use a photocell component, where the signal goes through a bulb, that turns on an off and thus the signal will be more of a square wave instead of a sine wave \citep{tremped}.  