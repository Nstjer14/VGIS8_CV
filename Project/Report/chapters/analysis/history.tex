\section{Effect Pedal History}
\label{sec:problem_analysis}
In the early 1930s the first amplified guitars started to appear. At this time big bands and horns were the stars and guitarists wanted to be so too. The problem was that the amplified guitar sound was very thin, which meant it did not bring the guitar sound to the front of the sound picture. This made guitarists and guitar manufacturers think of a way to make the guitar sound richer and be more noticeable. 

With the birth of the electric guitar came the first guitar effects. The first guitar effects were build in the guitar itself. The guitar manufacturer, Rickenbacker, started producing a guitar in the 1930s with a bar on the bridge, as seen in figure \ref{fig:rickenbacker_vibrola}, which made the bridge jiggle and by that made a vibrato effect and gave the guitarist a opportunity to vary the sound of the guitar.

\begin{figure}[htbp]
\centering
\includegraphics[width=0.5\textwidth]{Rick_SpanishVibrolaGold_B011.jpg}
\caption{The Rickenbacker Vibrola guitar \citep{rickenbacker_vibrola}.}
\label{fig:rickenbacker_vibrola}
\end{figure}

In the 1940 the company DeArmond made the first stand alone guitar effect, which was a tremolo effect. This made other companies and guitarists look for ways to make different audio effects. A lot of guitarists were looking for a way to create the effect of the echo and reverberation created in a big concert hall. The guitarist Duane Eddy made the first reverb effect for recording by connecting a speaker and a microphone to a 500 gallon water tank which created a natural reverb and echo effect. Later in the 1950s most amplifiers had built-in tremolo, vibrato, echo and reverb effects and in the 1960s the first tape based echo effects started to show up. These kind of effects were powered by vacuum tubes which made them expensive, fragile and not very practical for transportation. In the 1960s the transistor was introduced and was used instead of vacuum tubes to power the pedals. This made the pedals affordable and portable. because of the new technology the manufacturing exploded in the 1970s and created a whole new market for guitar effect pedals \citep{howguitarpedalsworks}.

As the digital technology quality increased in the late 1990s a new type of pedals started showing up. Because of the added noise when multiple analogue pedals are added in a chain, manufactures decided to make one pedal board with multiple effects. By doing this it is possible to change the order of the pedal chain and change settings of the individual pedals with a single switch without adding additional noise. This meant that the standard for noise in guitar pedals was increased which gave a less noisy output \citep{ashorthistoryofeffectpedals}.