%!TEX root = ../../master.tex
\section{Effect Description}\label{sec:effect_descr}
The analysis of the effects concerns several parameters of the effects and functionality of previous designs. This also includes explanations of the effect sound and what this is supposed to sound like.

\subsection{Delay Based Effects}
Delay is the act of holding a signal to be played back after a variable amount of time. 
Different lengths of delay combined with different couplings gives different effects. Flanger, chorus, echo and reverb are all examples of delay effects. These different delay based effects are defined as having a delay range, as seen in \autoref{tab:delay_times}. This means that all the different delays in the effect are in the delay range, in order to be defined as that specific effect.

\begin{table}[htbp]
\centering
\caption{Delay length for different kinds of effects \citep{DAFX}.}
\label{tab:delay_times}
\begin{tabular}{|l|l|l|}
\hline
\rowcolor{lightgray}
delay type   & delay length       & Effect type          \\ \hline
short delay  & \textless 15 ms    & flanger \\ \hline
medium delay & 15 ms - 50 ms      & chorus \\ \hline
long delay   & \textgreater 50 ms & echo, reverberation  \\ \hline
\end{tabular}
\end{table}

\subsection{Echo}
The echo effect is a delay based effect. Echo gives a repeating but decaying signal. This is implemented using a feedback loop with a delay as shown on \autoref{fig:echo_block_anal}. The delay part of the echo can be implemented with cascades of sample hold circuits, also known as a bucket brigade device, where the signal is sampled and sent through the system. The delay of the signal can be varied by changing the clocking frequency or the length of the bucket brigade. The clock is used to toggle the switches so the signal can move through the circuit. The switches open in sets where every other is opened while the rest is closed. Increased frequency gives a smaller delay while a shorter switching frequency gives a longer delay. The minimum sample frequency is set by the maximum frequency of the signal since it is converting the signal to discrete time in order to delay it \citep{se_bucket_brigade}. In \autoref{fig:bucketbrigade} a simple bucket brigade is illustrated.

\begin{figure}[htbp]
    \centering
    \includegraphics[width=0.8\textwidth]{block_echo}
    \caption{Block diagram of the echo effect.}
    \label{fig:echo_block_anal}
\end{figure}

\ctikzset{bipoles/length=1cm}
\begin{figure}[H]
    \centering
    \includegraphics[width=\textwidth]{bucketbrigade}
    \caption{A simple bucket brigade \citep{se_bucket_brigade}}{
    \label{fig:bucketbrigade}
    }
\end{figure}


Another way of implementing a delay is with two or more magnetic tape recorders with a shared tape. In this case the audio is recorded onto a tape loop with a magnetic write head. Further down the tape there is located one or more read heads which play the audio. The delay can be adjusted by changing the distance between the read and write heads or by changing the speed of the tape \citep{mag_tape_delay_hist}.

In order to turn these delays into a echo effect the original signal need to be added to the original signal. This can be done by using a voltage adder as shown in \autoref{fig:analog_echo}.

\begin{figure}[htbp]
    \centering
    \includegraphics[width=0.6\textwidth]{analog_echo}
    \caption{Implementation of analogue echo.}
    \label{fig:analog_echo}
\end{figure}


%!TEX root = ../../master.tex
\subsection{Flanger}
\label{ssub:flanger}

The flanger effect was introduced in the 1960's and was originally an analogue effect. The effect was made by using two tape machines playing two different tapes with the same music but in sync and the output was summed together. To create the effect a flange on one tape machine was touched lightly to create a small delay between the two tapes. The flange was then released and the same procedure was done on the other tape machine to counter the delay and create a delay on the other tape. The effect is often described as the sound of a jet plane passing by which makes the listener hear the direct sound from the plane but also a reflected sound from the ground beneath. 

Nowadays the flanger effect is mostly created digitally, as described in the block diagram in \autoref{fig:flanger_blockdiagram} and is modeled as a feed forward comb filter with a delay, $M$, which is varied over time. The output from the filter will then be as in \autoref{eq:flanger_out}.

\begin{figure}[htbp]
	\centering
	\includegraphics[width=0.8\textwidth]{figures/analysis/flanger_blockdiagram.pdf}
	\caption{Flanger block diagram.}
	\label{fig:flanger_blockdiagram}
\end{figure}

\begin{equation}
	y(n)=x(n)+g \cdot x(n-M(n))
	\label{eq:flanger_out}
\end{equation}

\startexplain
	\explain{$y(n)$ is the output over time n}{}
	\explain{$x(n)$ is the input signal}{}
	\explain{$g$ is the depth of the flanging effect}{}
	\explain{$M(n)$ is the length of the delay line at n samples}{}
\stopexplain

The delay line, $M(n)$, is typically varied according to a triangular or sinusoidal waveform and the delay length is modulated by a \gls{lfo}. The length of the delay is typically at a maximum of 15 ms as seen in \autoref{tab:delay_times}.  

The frequency response of the output is comb shaped as seen in \autoref{fig:comb_filter_response}.  

\begin{figure}[htbp]
	\centering
	\includegraphics[width=0.5\textwidth]{figures/analysis/comb_filter_math_response}
	\caption{Frequency response of a comb filter without scaling and numbers to visualize a comb filter.}
	\label{fig:comb_filter_response}
\end{figure}

For $g>0$ there are $M$ peaks in the frequency response which are centered around the frequencies calculated in \autoref{eq:comb_filter_freqs}.

\begin{equation}
\omega_{k}^{(p)}=k\cdot \frac{2\pi}{M}
\label{eq:comb_filter_freqs}
\end{equation}

\startexplain
\explain{k = 0,1,2,...,M-1}{}
\explain{M is the amout of notches}{}
\explain{p is the number of the specific peak}{}
\stopexplain

When $g =1$ the notches are at maximum attenuation with $M$ notches between the peaks at frequencies calculated in \autoref{eq:notch_freqs}.

\begin{equation}\label{eq:notch_freqs}
	\omega_k^{n}=\omega_k^{p}+\frac{\pi}{M}
\end{equation}

When $M$ varies the comb teeth squeezes in and out like the pleated layers on an accordion which produced the flanger effect as done in the 1960's with tape machines. \citep{flanger_descr}
%!TEX root = ../../master.tex
\section{Reverb}
\label{sec:design_reverb}
The reverb effect is simulated in MATLAB by following the block diagram in \autoref{fig:moorer_reverb_design} and the requirements set in \autoref{sec:reqDL}. The simulation is implemented with blocks instead of being sample based.

\subsection{General Algorithm}
The structure of the reverb effect can be visualised by the block diagram in \autoref{fig:moorer_reverb_design} which can be divided into three sections, a tapped delay line, an all-pass filter and comb filters which all contribute to different parts of the reverberation effect as described in \autoref{subs:reverb_analysis}. The different parts are simulated in MATLAB and combined to give the final effect with some variable that can be adjusted by the user. The variables are the pre delay which is the time from the direct sound to the first early reflection, the reverb time which is the time from the first reflection to the last reflection, the comb decay which controls how long it takes before the late reflections from the comb filter have decayed and the delay gain which controls how loud the reflections are and therefore the reverb effect is.

\begin{figure}[htbp]
	\centering
	\includegraphics[width=0.8\textwidth]{figures/analysis/moorers_reverb.pdf}
	\caption{Moorer’s reverberator.}
	\label{fig:moorer_reverb_design}
\end{figure}

\subsection{Simulation}

The simulation is made according to the block diagram in \autoref{fig:moorer_reverb_design}. First in the program the sample rate and the user variables are set as shown in \autoref{code:sim_reverb1}.


\begin{lstlisting}[caption={Sample rate and user variables.},language=MATLAB,label={code:sim_reverb1},tabsize=2]
Fs = 44100; %sample rate
len = Fs; 
x = 1:len;

%variables that can be changed by the user
preDelay = 50; 
reverbTime = 30;
combDecay = 0.01; %from 0 to 1
delayGain = 0.9; %from 0 to 1
\end{lstlisting}

\subsubsection{Tapped Delay Line}

In \autoref{code:sim_reverb_delayline} the tapped delay line is simulated which gives the early reflections as seen in \autoref{fig:reverb_impulse_response}. There are six reflections to simulate being inside a room with reflecting surfaces, ceiling, floor and four walls. In the delay line the \texttt{preDelay} can be adjusted to create a bigger delay before the first early reflection appears. The \texttt{reverbTime} can also be adjusted to make the time from the first early reflection to the last bigger, and by that make the overall reverb time bigger. The last variable in the delay line is the \texttt{delayGain} which specifies how loud the early reflections are compared to the direct signal.

\begin{lstlisting}[caption={Simulation of the tapped delay line.},language=MATLAB,label={code:sim_reverb_delayline},tabsize=2]
%delayline
len = length(sig);
out_dlyline = zeros(1,len);

dly_ms = preDelay + reverbTime .* [0 0.2 0.4 0.6 0.8 1]; %delay in ms
dlyline_samp = int32(dly_ms*10.^(-3)*Fs);
dlyline_gain = delayGain .* [1 0.9 0.8 0.7 0.6 0.5];

for n = 1:length(dlyline_samp)
    for i = 1:len
        if(i - dlyline_samp(n) > 0)
            out_dlyline(i) = out_dlyline(i) + dlyline_gain(n) * sig(i - dlyline_samp(n));
        end
    end
end
\end{lstlisting}

\subsubsection{Comb Filters}
As with the tapped delay line there are six comb filters to simulate a room with six reflections. The purpose of the comb filters is to simulate the late reflections which is the reflections of the reflections. These reflections quickly die out as shown in \autoref{fig:reverb_impulse_response} where the comb filters are creating the main reflections in the late reflection part including some extra small reflections which follows. The delay between these small reflections is distributed over a ratio of 1:1.5 between 50 and \SI{80}{\milli\second} as seen in the variable \textit{dly_ms} in \autoref{code:sim_reverb_comb} \citep{DAFX}. The comb filter is designed as described in \autoref{subs:reverb_analysis} and as seen in \autoref{fig:comb_iir1} which means that there is an $\alpha$ and a $\beta$ value.

\begin{figure}[htbp]
	\centering
	\includegraphics[width=0.8\textwidth]{figures/design/IIR_comb.pdf}
	\caption{IIR comb filter.}
	\label{fig:comb_iir1}
\end{figure}

The $\beta$ value is set to $\frac{1}{6}$ because there are six comb filters, and when the $\beta$ values are added together, it should be equal to one in total. The $\alpha$ value is calculated from \autoref{eq:reverb_alpha_value} \citep{DAFX}.

\begin{equation}\label{eq:reverb_alpha_value}
 	\alpha = 10^{-3\frac{T_d f_s}{m_i}}
\end{equation} 

\startexplain
	\explain{$\alpha$ is the attenuation coefficient, \texttt{comb_a} in the program}{\noSIunit}
	\explain{$T_d$ is the desired decay time, Td in the program}{\si{\milli \second}}
	\explain{$f_s$ is the sampling rate, Fs in the program}{\si{samples}/\si{sec}}
	\explain{$m_i$ is the delay length in samples, \texttt{comb_samp} in the program}{\si{samples}}
\stopexplain

Now that all the variables are known they are used to calculate the output of the comb filters which is done from line 13 in \autoref{code:sim_reverb_comb}. 

\begin{lstlisting}[caption={Simulation of the comb filters.},language=MATLAB,label={code:sim_reverb_comb},tabsize=2]
%comb filters
dly_ms = [50 56 60 64 69 75];
comb_samp = int32(dly_ms*10.^(-3)*Fs);
comb_b = [1/6 1/6 1/6 1/6 1/6 1/6];
Td = combDecay .* [11 15 17 20 24 26];
comb_a = [];
for n = 1:6
    comb_a = [comb_a (10.^((-3*Td(n)*10^-3*Fs)/double(comb_samp(n))))];
end

comb_out = zeros(1,len);
comb_in = [out_dlyline; out_dlyline; out_dlyline; out_dlyline; out_dlyline; out_dlyline];
for n = 1:6
    for i = 1:len
        if(i- comb_samp(n) > 0)
            comb_in(n,i) = comb_b(n) * comb_in(n,i) + comb_a(n) * comb_in(n,i - comb_samp(n));
        else
            comb_in(n,i) = comb_b(n) * comb_in(n,i);
        end
        comb_out(i) = comb_out(i) + comb_in(n, i);
    end
end
\end{lstlisting}

\subsubsection{All-pass Filter}
When the summed output of the comb filters is calculated they can be used as input in the all-pass filter. The all-pass filter has the purpose of making a muddy sounding effect to the reverb effect and by that making the effect deeper. To make the desired effect, the delay of the all-pass filter must be less than \SI{50}{\milli \second} and is chosen to be \SI{6}{\milli \second} \citep{DAFX}. The output of the comb filters and the delay time is put into the differential equation as seen in \autoref{code:sim_reverb_allpass}, which is then added to the tapped delay line output and gives the final output from the reverb effect. 

\begin{lstlisting}[caption={Simulation of the all-pass filter.},language=MATLAB,label={code:sim_reverb_allpass},tabsize=2]
%allpass
g = 0.7;
dly_samp = int32(6*10.^(-3)*Fs);

for i = 1:len
    if(i-dly_samp>0)
        out(i) = -g*comb_out(i) + comb_out(i-dly_samp) + g*out(i-dly_samp);
    else
        out(i) = -g*comb_out(i);
    end
end
\end{lstlisting}

The result of the simulation is shown in \autoref{fig:reverb_impulse_response} where the impulse response of the reverb effect is shown. 

\begin{figure}[htbp]
    \centering
    \includegraphics[scale=0.8]{reverb_freq}
    \caption{Impulse response of reverb.}
    \label{fig:reverb_impulse_response}
\end{figure}


%!TEX root = ../../master.tex
\section{Chorus}\label{sec:chorus_design}
\todo[author=Jacob]{Hele chorus afsnittet skal nok fjernes}
The chorus effect is made in regards to both the requirements and the analysis of the effect.

\subsection{Simulation}
The simulations and design of these are made with MATLAB.

The chorus effect works as shown in \autoref{fig:chorus_block}. This means the effect should have several feed forwards, giving the signal a delay and configuring the amplitude of the input signal.

When simulating this firstly a specific delay time is chosen. To add several delays, variations of the delay time are made by multiplying the delay with a value of $1.\text{xx}$. These delays are added to the original signal including an attenuation of the delays. 

A plot of the simulation is made with a single impulse to show the delays following the original signal. The x-axis shows the number of samples while the y-axis shows the amplification of the signal. This is shown in \autoref{fig:chorus_sim_impulse}.

\begin{figure}[htbp]
	\centering
	\includegraphics[width=0.8\textwidth]{chorus_sim_impulse}
	\caption{Impulse plot of simulation of the chorus effect.}
	\label{fig:chorus_sim_impulse}
\end{figure}

This is done with a sample rate of \SI{44.1}{\kilo\hertz}. By knowing the sample rate, it is possible to set the delay in seconds. In this occasion it is set at \SI{30}{\milli\second}.





\subsection{Distortion and Overdrive}\label{subs:dist_overdrive}
Distortion and overdrive are two different effects but still in the same category of amplitude effects. Overdrive is dependent of the amplitude and the amount of distortion will correspond to the amplitude of the input. Whereas distortion does not depend on the amplitude, and distort the signal to a level set by the user \citep{guitar_science2014}.

When distorting a signal the waveform of the signal is changed. The distortion effect is a hard clipping of the signal and the overdrive effect is a soft clipping of the signal. This is also illustrated in \autoref{fig:clipping_waveform}.

\begin{figure}[htbp]
    \centering
    \includegraphics[width=0.8\textwidth]{clipping_waveform.pdf}
    \caption{Illustration of soft and hard clipping of a signal \citep{clipping_fig}.}
    \label{fig:clipping_waveform}
\end{figure}

A simple block diagram of a distortion effect is made. This is illustrated in \autoref{fig:dist_block}.

\begin{figure}[htbp]
    \centering
    \includegraphics[width=0.6\textwidth]{distortion_block}
    \caption{A simple block diagram of a distortion effect.}
    \label{fig:dist_block}
\end{figure}

Both distortion and overdrive have a very lose definition, which is basicly that the signal is either clipped softly or hard. Most distortion and overdrive pedals are also analogue, and usually not implemented in a digital system. Therefore measurements on an analogue distortion pedal is made, to see the output compared to the input in one version of a distortion pedal. These measurements are shown in \autoref{cha:measuring_distortion_pedal_output}. 